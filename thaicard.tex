\documentclass[]{extarticle}
\usepackage[paperheight=15in,paperwidth=15in,margin=0.5in]{geometry}
\usepackage{tikz}
\usepackage{fontspec}
\usepackage{polyglossia}
\setdefaultlanguage{thai}
\setotherlanguages{english,russian}
%\setdefaultlanguage{russian}
%\setmainfont[Ligatures=TeX]{Times New Roman}
\newfontfamily\cyrillicfont{Tahoma Bold}[Script=Cyrillic]
\newfontfamily\thaifont{Sathu}[Script=Thai,Scale=MatchLowercase,WordSpace=1.25,Mapping=tex-text]
\usepackage{xunicode}
\usepackage{xltxtra}
\pagestyle{empty}
\begin{document}
\begin{tikzpicture}[remember picture,overlay]
% draw image
\node[inner sep=0] at (current page.center)
{\includegraphics[width=\paperwidth,height=\paperheight]{thaicard.png}};
\end{tikzpicture}
\resizebox*{\textwidth}{!}{คำศัพท์ วลี และประโยคต่างๆที่เกี่ยวกับ ``รูปลักษณ์''}
\hspace{0pt}
\vfill
\begin{center}
\resizebox*{\textwidth}{!}{\textrussian{\textbf{телослож\textcolor{red}{\'е}ние}}}
\vskip0.7in
\resizebox*{\textwidth}{!}{[ประเ-ภทข-\textcolor{red}{องร่า}-งกาย]}
\vskip0.7in
\resizebox*{\textwidth}{!}{\textbf{(ประเภทของร่า)งกาย}}
\end{center}
\vfill
\hspace{0pt}
\pagebreak
\end{document}
